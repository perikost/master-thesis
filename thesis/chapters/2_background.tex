\chapter{Background \& Related Work}\label{chapter:background}

\section{Ethereum}\label{sec:}
\section{Distributed File Systems}\label{sec:}
\subsection{IPFS}\label{subsection:}
\subsection{Swarm}\label{subsection:}
\subsection{Overview of IPFS and Swarm as decentralized storage solutions}\label{sec:}

\section{Related work - Purpose and significance of the thesis}\label{sec:}
There are tools for assisting developers to minimize gas consumption, but mostly, the focus is on the program structures rather than the data  \citep{nelaturu_2021, chen_2017, chen_2021}. Among them, we singled out Gasper  \citep{chen_2017}, a cost minimization tool that analyzes bytecode to determine costly patterns in SCs. Its creators used it to analyze approximately 4000 SCs and concluded that three of these patterns were found in the majority of them. This research was continued, and new patterns have been added  \citep{chen_2021}. There is also a work on parametric cost bounds in the Ethereum blockchain \citep{albert_2021}.

Data storage cost and gas saving methods are studied to some extent in the context of specific use cases  \citep{kurt_2020, delgado_2019, westerkamp_2020}. Gas consumption of SCs is examined in  \citep{grech_2020, signer_2018}, but none of them researches the related cost holistically by considering all possible ways of using Ethereum as a data store. In  \citep{consensys}, the use of events as an alternative data storage option was proposed. However, the novel idea of storing data in a transaction’s payload is barely studied. Blockchi \citep{yankov_2018} is a tool that allows storing and retrieving JSON objects, while in  \citep{xie_2017} a storage model is proposed for storing data from IoT sensors in hexadecimal format. In the latter, transaction references are retained in an external database. In our work, apart from evaluating the transaction payload as a data store, we propose an extension to this approach, which we term as \emph{``Unused Function Parameters``}. In short, it is a means of storing data in a transaction's payload while exploiting Solidity's built-in ABI interface to allow handling all available data types.

A common architecture for managing large chunks of data is to use distributed file systems such as IPFS and Swarm for storage and record the resulting content identifiers in Ethereum. This approach has been studied in several applications, e.g., \citep{hao_j_2018, ren_2021}, but no comments were made on the performance of these platforms nor on the cost of recording the content identifiers. Ramesh et al.\cite{ramesh_2019}, exploited such a scheme to handle IoT data and also studied both local retrieval and upload performances of IPFS and Swarm. In \citep{shen_2019}, IPFS was evaluated for remote and local retrieval latency. In \citep{abdullah_2021} the performance of IPFS in a private network was compared to that of FTP. Furthermore, a recent study conducted by Ismail et al. \cite{aisyah_2022} reviewed several works centered around the performance of distributed file systems. In light of the available literature, it is apparent that the majority of attention is being directed towards IPFS, leading us to believe that Swarm’s performance is an under-researched topic. On top of that, it's worth mentioning that research on Swarm's performance prior to three years ago should now be considered outdated, as a new client \citep{swarm_bee} and updates to the underlying protocols were introduced in 2020 \citep{tron_2020}. Our work is focused on local read-write performance of IPFS and Swarm, and the cost associated with storing the corresponding content identifiers in Ethereum.

In conclusion, even though the necessity of minimizing gas consumption is recognized in the literature, existing studies focus mainly on program structures or inefficient patterns. At the same time, efficient data management is an under-researched topic despite the fact that it is an essential part of DApps. In this work, through the research question we posed and the corresponding research we conducted, we intend to give a comprehensive perspective on efficient ways of handling data both on-chain and in hybrid architectures.

\subsection{Existing data management approaches in Ethereum-based dApps}\label{subsection:}
\subsection{Cost and retrieval performance of the aforesaid approaches}\label{subsection:}
\subsection{Cost and retrieval performance of the hybrid approaches}\label{subsection:}
\subsection{Purpose and significance of the thesis}\label{subsection:}

